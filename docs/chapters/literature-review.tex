Another cool thing about \LaTeX~is its referencing system. This template is set up to use harvard-style referencing. You can do this by using \texttt{\textbackslash citep\{citekey\}}. It will print out something like this: \citep{aad2012observation}. Or alternatively, you can use \texttt{\textbackslash cite\{citekey\}} to cite things like this: \cite{chatrchyan2012observation}. This template uses Bib\LaTeX~for referencing, with a Biber backend. This is primarily due to the extensive features Bib\LaTeX~provides, along with the option of glossaries. If you want to customise the referencing style, you can either modify the template slightly to use different options, or use \texttt{\textbackslash usepackage} again to reimport it. There's probably some commands to change its options after its been imported too.

\section{Ludography}
This thesis template also contains an optional ludography. This is primarily for Games Development students, who wish to cite games in their thesis. To use this, just put references into your bib file as usual with the game's details. Then, make sure \texttt{keywords} is set to \texttt{\{game\}}. This is what is used to determine which references are games, and which are actual papers. For a more elaborate example, see \texttt{bib/ludography.bib}.

Also, make sure that the \texttt{title} key is actually the author of the game, and the \texttt{author} is the title of the game. The reason this is swapped around is because Bib\LaTeX~likes to print references out with the author first. Then, just add \texttt{\textbackslash printLudography} with an optional title argument to print out all citations like \texttt{\textbackslash printLudography} or \texttt{\textbackslash printLudography[Games]}.

You can also use the \texttt{ludography} environment if you wish to print out some text before the list of games is printed. An example of this can be seen in \texttt{main.tex}. To cite games, you can \texttt{\textbackslash cite} it like any other reference. However, if you want it to display the title instead of the standard referencing style, you can use \texttt{\textbackslash citeGame} instead.

Here is an example of a cited game with a normal reference style: ~\cite{spaceinvaders}. Ugh, pretty ugly. Instead, here the two are  cited in the next sentence as games with \texttt{\textbackslash citeGame}. Both \citeGame{spaceinvaders} and \citeGame{breakout} were games made by Atari. Much better!