Accurate determination of relative poses is a key element in many robotic applications.\textit{Scan-Matching} is the method of registering two scans from perception sensors taken from two poses to determine the relative transformation between the two poses. It is the key step in successfully and efficiently solving \textit{Simultaneous Localization and Mapping} problem. The pose information available to the robot in motion are errorous and mapping with such information can result in inaccurate map of the environment.Various scan matching algorithms are widely used in many applications which are uniqely developed for a particular type of perception sensor. Generally, It corrects the pose information obtained from the motion sensors with the information obtained from the perception sensors\cite{GMap_algo}. With the advent of new sensors and improvement to the existing sensors, it is also possible to solve the \textit{Simultaneous Localization and Mapping} problem only with perception sensors \cite{ZhangS14}.

\textit{Simultaneous Localization and Mapping}, commonly abbreviated as \textit{SLAM} is one of the fundamental and essential task of any mobile robots which builds map of the environment without the knowledge of actual location and orientation of the robot. It is one of the key initial step for the development of navigation systems for any mobile robot. It is also referred as  the chicken or the egg problem, as it requires precise location of the robot in the environment, to get an accurate map of the environment, but in order to determine the precise location, the map of the environment is required. Given the complexity it is  harder to find a precise solution, accurate localization and mapping.

The complex task of localizing and mapping the surrounding has triggered a huge interest widely in the research filed and it continues to  improve as the technology evolves in both software and hardware. Over the past decade numerous stratergies and methods \cite{A.Pratik}, \cite{E.Olson_Map},\cite{T.seb_survey},\cite{D.Hahnel},\cite{M.Montemerlo},\cite{W.Burgard},\cite{GMap_algo}, \cite{G.Grisetti}, \cite{J.Kang},\cite{K.Konolige}, \cite{W.Burgard}, \cite{T.Reineking}, \cite{STWBDF}, \cite{D.Droeschel}, \cite{SUMA++}, \cite{VPS_SLAM}, \cite{RT_Graph},\cite{chen2020rss}, \cite{DBLP}, \cite{LOAM}, \cite{ChanWF18}, \cite{TatenoTLN17}, \cite{K.Ryu}, \cite{faucris.119665744}, \cite{Lu-2016-5572}, \cite{Mur_Artal_2015}, \cite{Mur_Artal_2017}, \cite{P.Agarwal}, \cite{segmap2018} were developed to solve the SLAM problem. 

A typical SLAM system uses a combination of odometry data, GPS, IMU to obtain information on the motion and Camera, LiDAR or Radar to observe the environment. The information obtained from these sensors are fused to obtain a full estimate of the state of the system. Though technologies evolved in acquiring precise measurement uncertainties and errors are still prevalent due to various factors. Hence designing a solution considering the uncertainties in the measurement is a key to achieve a good estimate of the location and map of the surrounding. Therefore a probabilistic approach is the key to globally consistent Map.

The SLAM problem can be classified in various ways.SLAM problem can be broadly classified into \textit{Online SLAM} and \textit{Full SLAM}. In the \textit{Online SLAM} problem, only the current state of the robot is estimated along with the map. In the \textit{Full SLAM} problem, the entire state of the path traversed by the robot is estimated.Based on the environment to be mapped, the SLAM system can be broadly classified into \textit{Feature-based SLAM} and \textit{Grid-based SLAM}. In \textit{Feature-based SLAM}, the robot looks for specific features or landmarks in the surrounding, the map obtained from it will specify the shape of the environment only at the landmark locations. However in real world scenarios, the robot should establish a correspondence between the expected landmark measurement and the actual landmark observed. The data association plays a key role in assigning the measurement to  its corresponding landmarks. With the known correspondence and known pose of the robot relative to a landmark, the problem reduces to Mapping with known pose. The \textit{Feature-based SLAM} is widely applied in many robotics application for instance, underwater exploration\cite{6678293}, mapping underground mines. These SLAM systems commonly use Ultrasonic sensors, camera for perception and GPS, odometry are used to find the motion of the robot.

In absence of landmarks, \textit{Grid-based SLAM} provides a complete SLAM solution with the help of 2-D or 3-D laser scanners or cameras. The observations made by the LiDAR are converted to an obstacle map also referred to as occupancy map. These observations are also used to correct the pose estimate of the robot. This SLAM approach provides a more useful way to map the surroundings as landmarks cannot be installed and observed in many challenging environment. Common approaches to \textit{Grid-based SLAM} are \textit{Particle Filters} and \textit{Graph Networks} More details on this approach will be provided in the chapter 4.

Numerous research has been done in the SLAM community. Some of the implementations are made available to the public in \textit{Robotic Operating system} framework,commonly abbreviated as \textit{ROS}. The most common are \textit{GMapping}\cite{gmap_ros} and \textit{Cartographer}\cite{cartographer_ros}.

This thesis presents insights on different scan-matching methods by applying it to Particle Filter based SLAM algorithm, namely \textit{gmapping}. These algorithms are implemented, tested and results are compared thoroughly to provide detailed report on throughput of the algorithms. The algorithm is tested on the data set obtained from \textit{CARLA}\cite{Dosovitskiy17}.\textit{CARLA} is an open-source simulator for autonomous driving research and development. It provides the environment for algorithm development, validation and testing. For more information on \textit{CARLA}, refer to \cite{Dosovitskiy17}. In this work  \textit{CARLA} is used extensively to gather data, which is used to implement various scan-matching algorithm and compare its performance .In the \textit{CARLA} simulator, an autonomous driving car mounted with sensors such as LiDAR, GNSS and IMU is driven through the simulated environment and all required data are collected in ROSBag through the ROS Bridge that establishes communication between  \textit{ROS} framework and \textit{CARLA}. The LiDAR is the perception system in the simulated car. It is a 360 degrees 3D LiDAR with 32 beams with a range of 50 meters rotating at a  frequency of 20 Hertz. The aforementioned sensors are placed at the same geometrical position on the vehicle and are oriented in the same direction which is also aligned to the vehicle coordinate system. The ISO sign convention is used for all the coordianted frames throughout this work and the convention defines  \textit{x}-axis(longitudinal axis) is pointing forward,  \textit{y}-axis (lateral axis) pointing left,  \textit{z}-axis (normal)pointing upward and the rotation is positive in the counter-clockwise direction. This avoids transforming measurements to different coordinate systems. The system is developed for an autonomous driving car trying to create a 2D Map of the simulated environment. Hence from this point, map generally refers to grid map, pose refers to Special Euclidean (SE2) and orientation, unless specified. 

 The thesis is organized as follows: The detailed literature review is provided on existing scan-matching methods, SLAM algorithms and applications, variants of SLAM in chapter 2. Probabilistic formulation of the scan-matching and SLAM problem is derived in chapter 3. Particle Filter based solution to SLAM is discussed and different implementations such as Rao-Blackwellized Particle Filter and GMapping algorithm are discussed in detail and analyzed in chapter 4. Scan-matching methods are discussed in detail in chapter 5.Comparison and bench-marking of the scan-matching methods are performed based on test results and computation is provided in chapter 6. Concluding remark on all the observations made so far in the paper is discussed in chapter 7.