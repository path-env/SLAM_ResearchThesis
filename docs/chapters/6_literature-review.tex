% General Literature introduction
The state of a robot such as position, orientation, velocity, acceleration, parameters or even a map, is critical to get accurate results. Needless to mention, the more information the robot knows about itself, the solution to find unknowns becomes easier. In the context of SLAM, pose information and the map of the environment are the states of absolute interest. Given the high precision pose information of the robot, then the problem simplifies to mapping of the environment with known pose.Here the state of interest is only the map that does not even require state estimation rather use the available mapping algorithm to draw the map of the environment. On the other hand, given the map of the environment, then the problem reduces to localization problem with data association in certain cases. Here the state of interest is pose and the correspondence for data association. A survey on various \textit{SLAM} methods were presented by \cite{C.Cadena} and \cite{T.Takleh}. According to \cite{C.Cadena}, the history of SLAM is broadly classified into \textit{classical age} consisting of \textit{Extended Kalman Filter (EKF)} methods, \textit{Rao-Blackwellized Particle Filter (RBPF)} methods and \textit{Maximum Likelihood Estimation}. Followed by \textit{algorithmic-analysis age} that predominantly studied observability, convergence and consistency of SLAM, leading to efficient SLAM solvers and many open source libraries. Followed by \textit{robust-perception age} that focuses on robust performance, resource awareness, high level understanding and task-driven perception. Finally in my perspective, with the development and up-rise in the application of Deep Learning techniques in SLAM, it would be suitable to term the current trend as \textit{Machine Learning era of SLAM}.

% SLAM - Kalman Filter based
In the beginning, hardware capabilities were very less and compute power embedded in mobile robots were limited to fewer processes at an instance. Hence the choice of sensors to be mounted on the robots were very critical. Historically the very first feasible solution to the SLAM problem proposed the use of Extended Kalman Filters(EKFs) by estimating the joint posterior distribution over pose of the robot and the landmark positions. Kalman Filters are widely used state estimators but are limited to linear system and works on the assumption of Gaussian distribution of data. Therefore they cannot be directly applied to solve the non-linear SLAM problems. By linearizing the data at the mean of the distribution it is possible to approximate the solution, the Extended Kalman Filter works on this principle. The state estimate and the associated uncertainty are provided by the mean and covariance matrix of the posterior of the joint probability. However the odometric drift observed by the robot seems to hold down the accuracy of the posterior estimate.\cite{J.Kang} proposed \textit{modified Neural Network aided EKF(mNNEKF)} to improve the accuracy of the estimate by compensating for odometric error of the robot using Neural Network. This method claims to work well even under colored noise and systematic bias error.

% SLAM - Particle Filter based
In order to model the underlying dynamics of the physical system, it is necessary to relax the assumption of linearity and Gaussian distribution of the data. This helps the system to be used for wide range of applications. \textit{Particle Filters} helps to model the required posterior distribution with the help of set of samples. As the number of samples increases, this representation becomes equivalent to the actual posterior distribution. \cite{S.Arulampalam} provides a detailed study on particle filters for nonlinear and non-Gaussian application and also performance measure of various particle filter algorithm. It is to be noted that the performance of the particle filters is primarily dependent on the proposal distribution from which particles are efficiently sampled and how re-sampling is carried over. \cite{GMap_algo} proposes how to model an accurate proposal distribution using the recent observation and better re-sampling strategies to avoid divergence of the filter. Detailed study on \textit{Particle Filters} is provided in chapter \ref{chap:Particle}.

% SM
Filters discussed above usually carry the state prediction and correction steps continuously on receiving new data. Though the motion models could help in determining the position of the robot to a level of confidence, it is susceptible to drifts and hence diverges. However these errors could be corrected in the state correction step with the help of observation model or scan matching algorithms using the data obtained from perception sensors. \cite{lu1997} proposed two methods to find the relative transformation using the 2D laser scanners data taken from two different poses. One by considering point to point correspondences between point clouds and finding the relative transformation using least squares. Another by matching point cloud data by tangent lines and minimizing the distance function between scan points. These methods are widely used in many applications, however they require good initialization for convergence.  \cite{D.Hahnel} developed techniques for map building even in populated environments using Sample-based Joint Probabilistic Data Association Filters (SJPDAFs). 
\par
\cite{Konolige} presented a method on using correlative scan-matching for Markov Localization(ML) using range scanners on a mobile robot and provides evidences on compute time efficiency compared to other methods. \cite{P.Biber} proposed an alternative method of representing the map using Normal Distributions Transform for any range scan and use it for finding relative transform between any two range scans. \cite{E.Olson/LocalSM} describes a robust place recognition algorithm that combines many uncertain local matches into very certain global match. The author also uses correlative scan-matching technique to search the 3D transformation parameter space to determine the correct translational and rotational components first from a coarse grid space, to determine approximate pose and then from a finer grid space, to determine the exact pose with a confidence. \cite{K.Ryu} combined the best of \cite{lu1997} and \cite{P.Biber} and proposed scan-to-map algorithm for accurate 2D map building. Finally \cite{7795620} studied time-related effects of scan sensors on grid based and object based approaches for stationary and movable elements. Detailed study on \textit{scan-matching} is provided in chapter \ref{chap:Scan}.

% SLAM - Graph Network based

% SLAM - Neural Network based

% SLAM - conclusion


%%%%%%%%%%%%%%%%%%%%%%%%%%%%%%%%%%%%%%%%%%%%%%%%%%%%%%%%%%%%%%%%%%%%%%%%%%%%%%%%%%%%%%%%%%%%%%%
% Another cool thing about \LaTeX~is its referencing system. This template is set up to use harvard-style referencing. You can do this by using \texttt{\textbackslash citep\{citekey\}}. It will print out something like this: \citep{aad2012observation}. Or alternatively, you can use \texttt{\textbackslash cite\{citekey\}} to cite things like this: \cite{chatrchyan2012observation}. This template uses Bib\LaTeX~for referencing, with a Biber backend. This is primarily due to the extensive features Bib\LaTeX~provides, along with the option of glossaries. If you want to customise the referencing style, you can either modify the template slightly to use different options, or use \texttt{\textbackslash usepackage} again to reimport it. There's probably some commands to change its options after its been imported too.

% \section{Ludography}
% This thesis template also contains an optional ludography. This is primarily for Games Development students, who wish to cite games in their thesis. To use this, just put references into your bib file as usual with the game's details. Then, make sure \texttt{keywords} is set to \texttt{\{game\}}. This is what is used to determine which references are games, and which are actual papers. For a more elaborate example, see \texttt{bib/ludography.bib}.

% Also, make sure that the \texttt{title} key is actually the author of the game, and the \texttt{author} is the title of the game. The reason this is swapped around is because Bib\LaTeX~likes to print references out with the author first. Then, just add \texttt{\textbackslash printLudography} with an optional title argument to print out all citations like \texttt{\textbackslash printLudography} or \texttt{\textbackslash printLudography[Games]}.

% You can also use the \texttt{ludography} environment if you wish to print out some text before the list of games is printed. An example of this can be seen in \texttt{main.tex}. To cite games, you can \texttt{\textbackslash cite} it like any other reference. However, if you want it to display the title instead of the standard referencing style, you can use \texttt{\textbackslash citeGame} instead.

% Here is an example of a cited game with a normal reference style: ~\cite{spaceinvaders}. Ugh, pretty ugly. Instead, here the two are  cited in the next sentence as games with \texttt{\textbackslash citeGame}. Both \citeGame{spaceinvaders} and \citeGame{breakout} were games made by Atari. Much better!