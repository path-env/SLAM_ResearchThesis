\begin{abstract}
   Scan matching is the process of aligning two scans taken at two different but relatively close but unknown poses in order to obtain the relative transformation between the two poses. It has been widely used in the field of robotics especially in mapping applications. Simultaneous Localization and Mapping (SLAM) is a fundamental capability of any mobile robot that can map the surrounding as it drives by estimating the correct pose even in the presence of incorrect sensor measurements and observations. SLAM systems are prevalent in many robots available the markets such as robotic vacuum cleaners, automatic lawn movers, robotic door delivery machines... However it is continuously studied and improved with advancing sensor and computing technologies. Almost all modern day SLAM systems use scan matching algorithms for mapping as robots are equipped with high precision perception sensors that can provide more accurate readings of the surrounding.
   \par
   Various scan matching algorithms are widely used in the literature and the industry with advantages and disadvantages. Hence in this work various scan matching algorithms are studied, implemented and applied to particle filter based SLAM system(gMapping). The implementation is built from ground up without the use of any SLAM packages. The implemented algorithms are then evaluated using a dataset recorded from a simulated environment (CARLA). The results are then compared and a conclusive study is provided on the observations.
\end{abstract}